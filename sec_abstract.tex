\begin{abstract}

Smartphones and tablets have become the devices of choice for connecting to the internet and watching videos. Video data contributes to the bulk of internet traffic and the number, length and variety of videos have increased significantly. 
Mobile devices live on limited battery energy which is still a major bottleneck and a source of user dissatisfaction while watching videos. 
Radio energy contributes to as much as 40\% of the overall energy consumed 
by the device for many video streaming applications.
% and increases with throughput of the network for LTE and 3G. 
Thus a better understanding of energy efficiency of various streaming protocols and applications, used today, is need for future design of more energy efficient protocols for video streaming over cellular and .   
%In this paper we introduce an intermediate framework called \myname for power efficient video delivery to smartphone and tablets.
%This almost transparent battery aware framework takes away some of the video processing overhead from the device and intelligently tunes its parameters %customized for the mobile device while delivering the video using a novel transport protocol. Our preliminary results show that this framework can %significantly reduce energy consumption upto \texttt{50\%} of a mobile device without compromising user experience.  
In this paper we characterize the power efficiency of popular mobile video streaming services such as \textit{YouTube}, \textit{Netflix} and \textit{hulu}, over 4G . We also aim to identify components where power efficiency can be improved.
%We evaluate the energy efficiency of various streaming applications like \textit{YouTube}, \textit{Netflix} and \textit{hulu} and aim to identify components where power efficiency can be improved. 
% subrata: we normally do not use references in the abstract
%We collect packet traces for these streaming applications and pinpoint source of wasteful data and power drains with the help of ARO\cite{qian2011demo}, a widely accepted tool developed by AT\&T researchers for power modelling.  
We collect packet traces for these streaming applications and pinpoint source of wasteful data and power drains using power modelling tools.     
\end{abstract}

%\vspace{-0.1in}
%\noindent

%\keywords{
%Power efficiency, video streaming, transcoding, smartphone, tablet, offloading  


%\vspace{-0.2in}
%\noindent