\section{Related Work}
%mod test-sambit
A lot of research has been done in improving the power efficiency of mobile and hand-held devices. Different schemes target different features of the device for power reduction. These include network, architecture level, circuit, and software level optimizations. ~\cite{mohapatra_mm03} explores a combination of these optimizations for achieving power efficiency. Some recent literature like ~\cite{mobiarch,movid} also confirm the theory that transmitting video frames in bursts actually save significant amount of energy. A very recent work ~\cite{ucsd_wcnc} uses battery aware rate adaptation and base station reconfiguration to achieve power savings. An interesting work which effectively schedules data transfers in return for energy savings is~\cite{balasubramanian_imc09}. There are also studies which try to look at the energy consumption of web based applications in smartphones. Huang et al.,\cite{huang2012close} investigate the energy usage in 3G, LTE and WiFi networks and perform case studies of several popular applications on Android in LTE and identify that performance bottlenecks lies less in the network and more in the device's processing power.
Huang et al.,\cite{huang2013depth} conducted an in-depth study of interactions among applications, network transport protocol, and the radio layer and their impact on performance, using a combination of active and passive measurements. They discovered that many TCP connections significantly under-utilize the available bandwidth.This causes data downloads to be longer, and incur additional energy overhead.  
Deng and Balakrishnan\cite{deng2012traffic} propose a technique to reduce energy consumption by learning the traffic patterns and predicting when a burst of traffic will start or end and then determine when to change the radio's state from Active to Idle, and another to change the radio's state from Idle to Active.
Hoque, Siekinnen and Nurminen\cite{hoque2013using} propose a download scheduling algorithm based on crowd-sourced video viewing statistics which judiciously evaluates the probability of a user interrupting a video viewing in order to perform the right amount of prefetching. They show an energy savings of upto 80%.
However the problem still persists. 

