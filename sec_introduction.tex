\section{Introduction}
Video content constitutes a dominant fraction in internet traffic today. Trends show that this is prone to increase in the next few years~\cite{}.
Users prefer mobile devices for video streaming ~\cite{}, fueled by the increased throughput and reduced latency of mobile networks (LTE performs similar to Wi-Fi ~\cite{LTEPaper}).
Moreover, the cost of content delivery is also reducing due to ease of manageability and increased return of investment. As a result, many video content providers are increasing their mobile footprint in the recent years (e.g. Netflix, Youtube, Hulu). 
They follow a subscription based cost model and in order to improve user engagement, they provide more attention towards optimizing user experience. The video streaming services adapt video delivery according to the client device and network conditions by adapting the bit rate of the content sent to the client and delivering the content in smaller chunks. 
More preference is given to HTTP than specialized video streaming protocols due to its ease of adaptability in practice.  
DASH is a popular protocol used by video content providers to adapt bit rate of the video used by the client. Despite many such efforts, a common case of user frustration is the battery drain while watching videos and it is more pronounced in LTE owing to the cellular radio interface. 
 
In LTE networks, the Radio Resource Control (RRC) state machine ~\cite{LTEPaper} critically impacts the performance and power profiles of aplications. Various inactivity timers are used by network operators for saving radio power consumption by trading-off packet latency. Figure ~\ref{fig:LTERRCState} shows the RRC states in LTE. LTE has two RRC states - RRC\_CONNECTED and RRC\_IDLE depending on whether data is transferred or not. The RRC\_CONNECTED state has power conserving states including Short and Long DRX. In the DRX states the radio periodically polls the network for data and is idle otherwise. But the power consumed in the DRX states are still higher than that in the RRC\_IDLE state. 


Applications should intelligently schedule data transfer to obtain the best possible performance, energy and cost savings. Relevant metrics for video streaming - buffering rate, join time, playback time. Radio consumes significant device power. For better user experience, limit the join time and playback time by buffering aggressively. But to save on cost the application should buffer sparsely. Optimizing energy consumption requires in-depth understanding of the interaction of the application with the RRC state transition.Existing services aim at improving performance - esp reducing the playback time. Some of the popular streaming services use DASH - Dynamic adaptive streaming using HTTP. Video is sent in chunks depending on user interaction and buffer drain rate. It is not clear whether such adaptive bit rate approach, chunking would provide energy savings as they are not aware of the RRC state transitions.           

In this work we perform study to characterize the power characteristics of video streaming using popular streaming services like Netflix, Youtube and Hulu. The study is especially important and relavant in the context of mobile networks - 3G and 4G inorder to design cellular-friendly streaming services. The success of the project lies on jointly optimizing energy, user experience and data cost for video streaming in LTE, 3G.      
